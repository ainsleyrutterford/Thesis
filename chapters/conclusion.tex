\section{Contributions and Achievements}

\lipsum[1]
\lipsum[2]
\lipsum[3]

\section{Final Status}

2D architecture achieves good results that enable some sort of semi-automated calcification estimation via the network and the calcification scripts. 3D does not. The tools have been packaged as executables that can be easily used by researchers and are publicly available at GitHub (footnote). 

\lipsum[4]
\lipsum[5]

\section{Future Work}

Train with larger datasets. Relabel the 3D data to be more consistent and surely the 3D data would help performance. can look at corallites in neighbouring slices and can see that that although the corallites do not span more than ${\sim}5$ pixels, the annual density bands span up to 40 pixels in width. Would reduce the amount of mistakes involving the corallites.

Proper working 3D is probs a good future work

\lipsum[6]

\subsubsection{Further Possible Experiments}

Spatial Dropout?

L2 Regularization

Batch Normalization

% {\bf A compulsory chapter,     of roughly $5$ pages} 
% \vspace{1cm} 

% \noindent
% The concluding chapter of a dissertation is often underutilised because it 
% is too often left too close to the deadline: it is important to allocation
% enough attention.  Ideally, the chapter will consist of three parts:

% \begin{enumerate}
% \item (Re)summarise the main contributions and achievements, in essence
%       summing up the content.
% \item Clearly state the current project status (e.g., ``X is working, Y 
%       is not'') and evaluate what has been achieved with respect to the 
%       initial aims and objectives (e.g., ``I completed aim X outlined 
%       previously, the evidence for this is within Chapter Y'').  There 
%       is no problem including aims which were not completed, but it is 
%       important to evaluate and/or justify why this is the case.
% \item Outline any open problems or future plans.  Rather than treat this
%       only as an exercise in what you {\em could} have done given more 
%       time, try to focus on any unexplored options or interesting outcomes
%       (e.g., ``my experiment for X gave counter-intuitive results, this 
%       could be because Y and would form an interesting area for further 
%       study'' or ``users found feature Z of my software difficult to use,
%       which is obvious in hindsight but not during at design stage; to 
%       resolve this, I could clearly apply the technique of Smith [7]'').
% \end{enumerate}

% =============================================================================

% Finally, after the main matter, the back matter is specified.  This is
% typically populated with just the bibliography.  LaTeX deals with these
% in one of two ways, namely
%
% - inline, which roughly means the author specifies entries using the 
%   \bibitem macro and typesets them manually, or
% - using BiBTeX, which means entries are contained in a separate file
%   (which is essentially a databased) then inported; this is the 
%   approach used below, with the databased being dissertation.bib.
%
% Either way, the each entry has a key (or identifier) which can be used
% in the main matter to cite it, e.g., \cite{X}, \cite[Chapter 2}{Y}.
