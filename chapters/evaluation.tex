\section{2D Experimentation}

Need to talk about dropout? maybe only mention it in evaluation?
Spatial drop out better for fully conv? \url{https://arxiv.org/pdf/1411.4280.pdf}

% From Keras docs I think
% This version performs the same function as Dropout, however it drops
% entire 2D feature maps instead of individual elements. If adjacent pixels
% within feature maps are strongly correlated (as is normally the case in
% early convolution layers) then regular dropout will not regularize the
% activations and will otherwise just result in an effective learning rate
% decrease. In this case, SpatialDropout2D will help promote independence
% between feature maps and should be used instead.

\subsection{Augmentation}
\label{sec:evalaugmentation}

\subsection{Hyperparameter optimisation}

\subsection{Ablation}

comparison of convtranspose to upsample and then conv.

remove the relu at the end of the implementation?

remove the 

\section{3D Experimentation}

\section{Final Results}

\section{Comparison with other architectures?}

Compare SegNet, U-Net, pix2pix. Maybe find somewhere to put in the pix2pix generation for fun? If not don't worry. Do mention pix2pix using different backbones or whatever its called and why you think it didnt perform well.

\section{Cross-validation}
\label{sec:evalcrossval}

\section{Critical Evaluation}

\section{Comparisons with Existing Techniques}

\section{Future Work}

% {\bf A topic-specific chapter, of roughly $15$ pages} 
% \vspace{1cm} 

% \noindent
% This chapter is intended to evaluate what you did.  The content is highly 
% topic-specific, but for many projects will have flavours of the following:

% \begin{enumerate}
% \item functional  testing, including analysis and explanation of failure 
%       cases,
% \item behavioural testing, often including analysis of any results that 
%       draw some form of conclusion wrt. the aims and objectives,
%       and
% \item evaluation of options and decisions within the project, and/or a
%       comparison with alternatives.
% \end{enumerate}

% \noindent
% This chapter often acts to differentiate project quality: even if the work
% completed is of a high technical quality, critical yet objective evaluation 
% and comparison of the outcomes is crucial.  In essence, the reader wants to
% learn something, so the worst examples amount to simple statements of fact 
% (e.g., ``graph X shows the result is Y''); the best examples are analytical 
% and exploratory (e.g., ``graph X shows the result is Y, which means Z; this 
% contradicts [1], which may be because I use a different assumption'').  As 
% such, both positive {\em and} negative outcomes are valid {\em if} presented 
% in a suitable manner.