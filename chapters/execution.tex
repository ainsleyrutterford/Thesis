\section{Two Dimensional Boundary Extraction}

\subsection{Labelling the Data}

\subsection{Dataset Curation}

\subsection{Sliding Window}

\subsubsection{Splitting the Dataset}

\subsection{Architecture}

\subsection{Data Augmentation}

\subsection{Early Stopping and Checkpointing}

Results?

Compare SegNet, U-Net, pix2pix. Maybe find somewhere to put in the pix2pix generation for fun? If not don't worry. Do mention pix2pix using different backbones or whatever its called and why you think it didnt perform well.

\section{Three Dimensional Boundary Extraction}

\subsection{Dataset Expansion}

\subsection{Architecture Modification}

\subsection{Data Loader Implementation}

\subsection{Three Dimensional Data Augmentation}

\section{Accuracy Metric Implementation}

\subsection{Skeletonisation}

(Show algorithm and maybe some source code?)

\subsection{The \texttt{ctypes} Module}

\section{Density Band Width Estimation}

\subsection{Point Sampling}

% {\bf A topic-specific chapter, of roughly $15$ pages} 
% \vspace{1cm} 

% \noindent
% This chapter is intended to describe what you did: the goal is to explain
% the main activity or activities, of any type, which constituted your work 
% during the project.  The content is highly topic-specific, but for many 
% projects it will make sense to split the chapter into two sections: one 
% will discuss the design of something (e.g., some hardware or software, or 
% an algorithm, or experiment), including any rationale or decisions made, 
% and the other will discuss how this design was realised via some form of 
% implementation.  

% This is, of course, far from ideal for {\em many} project topics.  Some
% situations which clearly require a different approach include:

% \begin{itemize}
% \item In a project where asymptotic analysis of some algorithm is the goal,
%       there is no real ``design and implementation'' in a traditional sense
%       even though the activity of analysis is clearly within the remit of
%       this chapter.
% \item In a project where analysis of some results is as major, or a more
%       major goal than the implementation that produced them, it might be
%       sensible to merge this chapter with the next one: the main activity 
%       is such that discussion of the results cannot be viewed separately.
% \end{itemize}

% \noindent
% Note that it is common to include evidence of ``best practice'' project 
% management (e.g., use of version control, choice of programming language 
% and so on).  Rather than simply a rote list, make sure any such content 
% is useful and/or informative in some way: for example, if there was a 
% decision to be made then explain the trade-offs and implications 
% involved.

% \section{Example Section}

% This is an example section; 
% the following content is auto-generated dummy text.
% \lipsum

% \subsection{Example Sub-section}

% \begin{figure}[t]
% \centering
% foo
% \caption{This is an example figure.}
% \label{fig}
% \end{figure}

% \begin{table}[t]
% \centering
% \begin{tabular}{|cc|c|}
% \hline
% foo      & bar      & baz      \\
% \hline
% $0     $ & $0     $ & $0     $ \\
% $1     $ & $1     $ & $1     $ \\
% $\vdots$ & $\vdots$ & $\vdots$ \\
% $9     $ & $9     $ & $9     $ \\
% \hline
% \end{tabular}
% \caption{This is an example table.}
% \label{tab}
% \end{table}

% \begin{algorithm}[t]
% \For{$i=0$ {\bf upto} $n$}{
%   $t_i \leftarrow 0$\;
% }
% \caption{This is an example algorithm.}
% \label{alg}
% \end{algorithm}

% \begin{lstlisting}[float={t},caption={This is an example listing.},label={lst},language=C]
% for( i = 0; i < n; i++ ) {
%   t[ i ] = 0;
% }
% \end{lstlisting}

% This is an example sub-section;
% the following content is auto-generated dummy text.
% Notice the examples in Figure~\ref{fig}, Table~\ref{tab}, Algorithm~\ref{alg}
% and Listing~\ref{lst}.
% \lipsum

% \subsubsection{Example Sub-sub-section}

% This is an example sub-sub-section;
% the following content is auto-generated dummy text.
% \lipsum

% \paragraph{Example paragraph.}

% This is an example paragraph; note the trailing full-stop in the title,
% which is intended to ensure it does not run into the text.
