% {\bf A compulsory section, of at most $1$ page} 
% \vspace{1cm} 

% \noindent
% This section should pr\'{e}cis the project context, aims and objectives,
% and main contributions (e.g., deliverables) and achievements; the same 
% section may be called an abstract elsewhere.  The goal is to ensure the 
% reader is clear about what the topic is, what you have done within this 
% topic, {\em and} what your view of the outcome is.

% The former aspects should be guided by your specification: essentially 
% this section is a (very) short version of what is typically the first 
% chapter.  Note that for research-type projects, this {\bf must} include 
% a clear research hypothesis.  This will obviously differ significantly
% for each project, but an example might be as follows:

% \begin{quote}
% My research hypothesis is that a suitable genetic algorithm will yield
% more accurate results (when applied to the standard ACME data set) than 
% the algorithm proposed by Jones and Smith, while also executing in less
% time.
% \end{quote}

% \noindent
% The latter aspects should (ideally) be presented as a concise, factual 
% bullet point list.  Again the points will differ for each project, but 
% an might be as follows:

% \begin{quote}
% \noindent
% \begin{itemize}
% \item I spent $120$ hours collecting material on and learning about the 
%       Java garbage-collection sub-system. 
% \item I wrote a total of $5000$ lines of source code, comprising a Linux 
%       device driver for a robot (in C) and a GUI (in Java) that is 
%       used to control it.
% \item I designed a new algorithm for computing the non-linear mapping 
%       from A-space to B-space using a genetic algorithm, see page $17$.
% \item I implemented a version of the algorithm proposed by Jones and 
%       Smith in [6], see page $12$, corrected a mistake in it, and 
%       compared the results with several alternatives.
% \end{itemize}
% \end{quote}

This project is concerned with the use of deep learning to automate the estimation of the amount of annual skeletal matter produced by corals.

First, the process of labelling density bands present in two dimensional slices is automated. This is achieved using a convolutional neural network based off of the U-Net architecture~\cite{ronneberger2015u}. To train this network, a dataset containing ${\sim}600$ images was curated from slices extracted from seven individual coral samples. Various data augmentation techniques were also used to artificially expand the dataset. To measure performance, an accuracy metric based off of the Hausdorff distance between the predicted and ground truth boundaries is used. This initial architecture achieved an average Hausdorff distance of 1.X pixels.

Second, the process of labelling density bands present in three dimensions is explored. To this end, the U-Net architecture is modified to process three dimensional data. A larger dataset of manually labelled adjacent slices was curated to train this network containing ${\sim}X$ images extracted from $X$ coral samples.

Finally, once the density bands are identified, the estimation of the distance between the bands is automated using various techniques. These distances are then used to automatically estimate the volume of skeletal matter produced annually.

\section*{Summary of Achievements}

\begin{itemize}
    \item I spent 25 hours manually labelling coral data.
    \item I generated a 2D dataset containing ${\sim}600$ images and a 3D dataset containing ${\sim}X$ images using various scripts I wrote.
    \item Having not taken the ``Deep Learning'' unit, I spent 20 hours researching and learning about the field of deep learning.
    \item I learned how to use the Keras and TensorFlow libraries in order to implement the networks used.
    \item I modified an existing 2D U-Net implementation and trained it on the curated dataset.
    \item I extended the U-Net architecture to process 3D data.
    \item I wrote a ``data generator'' from scratch in Python to allow the input of 3D data into the network since Keras does not have its own implementation.
    \item I implemented online data augmentation of the 3D data.
    \item I implemented a custom accuracy metric in C and wrapped it in a Python function to achieve a ${\sim}625$ times speed up allowing the accuracy to be calculated in minutes rather than days.
    \item I performed hyper parameter optimisation in order to maximise the performance achieved by the various architectures.
\end{itemize}