Various supporting technologies such as software packages, libraries, and computing resources were used throughout the project. These technologies are outlined below.

\vspace{0.5cm} 

\begin{itemize}
\item Avizo\footnote{\url{https://tiny.cc/avizo}}: the Avizo software package is a 3D visualisation program and was used to open and view the 3D CT data interactively. It was vital in the selection of the appropriate slices to label and use to create the dataset.
\item GIMP\footnote{\url{https://www.gimp.org}}: the GNU Image Manipulation Program. GIMP is a free open-source cross-platform image editor and was used to manually label slices once they were extracted.
\item Keras\footnote{\url{https://keras.io}}: the Keras library is an open-source deep learning library written in Python. All architectures experimented with throughout the project were implemented using Keras.
\item TensorFlow\footnote{\url{https://tensorflow.org}}: the Keras library makes use of a TensorFlow backend to allow the code to run on both CPUs and GPUs. The TensorFlow library was also utilised in order to implement the focal loss function~\cite{focalloss} experimented with throughout the project.
\item OpenCV\footnote{\url{https://opencv.org}}: the OpenCV library is an open-source computer vision library. Although many Python modules were used throughout the project, the OpenCV module was the main module used to read, write, and manipulate any images.
\item BlueCrystal Phase 4\footnote{\url{https://www.acrc.bris.ac.uk/acrc/phase4.htm}}: in the early stages of the project, training and testing relied heavily on the GPU nodes of the BlueCrystal Phase 4 supercomputer.
\item GW4 Isambard\footnote{\url{https://gw4.ac.uk/isambard}}: ultimately, the majority of training and testing took place on the phase 1 GPU nodes of the Isambard supercomputer.
\end{itemize}