\begin{lstlisting}[float={t},caption={A simplified example of online augmentation implemented using the \texttt{ImageDataGenerator} class. The model is then trained for five epochs with each epoch consisting of 1000 training samples.},label={lst:augment},language=Python,upquote=true]
from keras.preprocessing.image import ImageDataGenerator

# Specify the transformations allowed and store them in a dictionary
aug = dict(rotation_range=2,
           width_shift_range=0.02,
           height_shift_range=0.02,
           shear_range=2,
           zoom_range=0.02,
           brightness_range=[0.9,1.1],
           horizontal_flip=True,
           vertical_flip=True,
           fill_mode='nearest')

# Pass the contents of the dictionary as arguments to the ImageDataGenerator
# constructors
image_datagen = ImageDataGenerator(**aug)
label_datagen = ImageDataGenerator(**aug)

# Create the generators using the flow_from_directory method. The same seed
# value is passed to both methods ensuring the same transformations are applied.
image_generator = image_datagen.flow_from_directory(image_path, seed=seed, ...)
label_generator = label_datagen.flow_from_directory(label_path, seed=seed, ...)

# Zip the generators into one generator that yields image-label pairs
train_generator = zip(image_generator, label_generator)

# Fit the model using the zipped generator
model.fit_generator(train_generator, steps_per_epoch=1000, epochs=5, ...)
\end{lstlisting}